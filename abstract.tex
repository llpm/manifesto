% !TEX root = llpm_manifesto.tex

\begin{abstract}

Despite a variety of promising work in hardware design methodologies, the
existing ecosystem of hardware design tools is broken. Today's
state-of-the-practice design toolsets are comprised of poorly-written
proprietary software cobbled together via brittle scripts written by
end-users. Further, RTL languages like Verilog are utilized as de facto
intermediate representations, though significant quantities of important
semantic information cannot be encoded in such low level representation. As a
direct result, both hardware design tasks themselves and innovating on new
hardware design methods tends to be extremely difficult.

To remedy many of the problems in hardware design, we propose the LLPM (Low
Level Physical Machine) framework. LLPM is intended to provide for hardware
what LLVM~\cite{llvm} has for software. Specifically, LLPM serves as
middleware for hardware compiler flows. It defines an intermediate
representation (IR) for hardware which can be produced by language or design
front ends, transformed by optimization passes, and synthesized into ASICs,
FPGA bitcode, RTL, or software simulators. LLPM's open framework and IR makes
it easier for new design/test/verification techniques, languages, and tools to
be implemented within a production framework. Should an ecosystem evolve
around it, LLPM will enable easier hardware design and debugging.

In this manifesto we introduce LLPM, both its IR and software framework. We
define the LLPM IR, which is a strongly-typed, latency-insensitive,
multi-granular dataflow graph. To both motivate the IR and set a roadmap for
LLPM's development, we also discuss a number of potential optimizations which
can be implemented in LLPM as well as potential front and back ends. Finally,
LLPM is intended to be flexible and extensible, enabling the integration of
design automation and optimization. As such, also we outline a proposal for
the design of LLPM as a software framework and tool flow.

\end{abstract}